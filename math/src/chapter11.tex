\section*{Chapter 11: Permutation Algorithms}

\paragraph{Exercise 11.1}
Prove \textsl{Cayley's theorem}: Any group $G$ is isomorphic to a subgroup of
the symmetric group on $G$, \Sym{G}.

\begin{proof}

For any $a \in G$, consider the following function $F_a: G \rightarrow G$:
$$F_a(x) = a x$$
\begin{itemize}
    \item $F_a$ is one-to-one, since $F_a(x) = a x = a y = F_a(y)$ implies 
    that $x = y$ (left multiplying by \Inv{a}).
    \item $F_a$ is onto: for a given $y \in G$, $F_a(\Inv{a} y) = a (\Inv{a} y)
    = (a \Inv{a}) y = y$.
\end{itemize}
Thus, $F_a$ is a bijection on $G$, which in turn means that $F_a$ is a permutation
of the elements in $G$. Thus, $S = \{ F_a \, / \, a \in G \}$ is a subset of \Sym{G}.
Moreover, $S$ is a subgroup of \Sym{G}:
\begin{itemize}
    \item $S$ contains the identity permutation: $F_{\Id{G}}(x) = \Id{G} \, x = x$.
    \item $S$ is closed by composition: $(F_a \circ F_b)(x) = F_a(F_b(x)) = a (b x)
    = (a b) x = F_{ab}(x)$.
    \item $S$ is closed by inverses: $\Inv{F_a} = F_{\Inv{a}}$, since
    $(F_a \circ F_{\Inv{a}})(x) = a (\Inv{a} x) = x$.
\end{itemize}
Let $\mathcal{F} : G \rightarrow S$  be a function defined as follows:
$$\mathcal{F}(a) = F_a$$
Then, $\mathcal{F}$ is a group isomorphism from $G$ to $S$:
\begin{itemize}
    \item $\mathcal{F}(ab) = F_{ab} = F_a \circ F_b =
    \mathcal{F}(a) \circ \mathcal{F}(b)$, using that $S$ is closed by composition.
    \item $\mathcal{F}$ is one-to-one, since $\mathcal{F}(a) = \mathcal{F}(b)$
    implies that $F_a(x) = a x = b x = F_b(x)$, and so $a = b$ after right
    multiplying by \Inv{x}.
    \item $\mathcal{F}$ is onto, since for a given $F_a \in S$, $\mathcal{F}(a) = F_a$.
\end{itemize}
\end{proof}

\paragraph{Exercise 11.2}
What is the order of $S_n$?

\begin{proof}[Answer]
$S_n$ contains every permutation of $\{1,\dots,n\}$. Since there are $n!$ of
them, the order of $S_n$ is $n!$.
\end{proof}

\paragraph{Exercise 11.3}
Prove that, if $n > 2$, $S_n$ is not abelian.

\begin{proof}
Let $P_1 = (2 \, 1 \, 3 \, \dots \, n)$ and let
$P_2 = (3 \, 1 \, 2 \, \dots \, n)$. Then,
\begin{itemize}
    \item $P_1 \circ P_2 = (1 \, 3 \, 2 \, \dots \, n)$, and
    \item $P_2 \circ P_1 = (3 \, 2 \, 1 \, \dots \, n)$.
\end{itemize}
Since $P_1 \circ P_2 \neq P_2 \circ P_1$, $S_n$ is not commutative.
\end{proof}


\paragraph{Exercise 11.9}
Prove that if a rotation of $n$ elements has a trivial cycle, then it has $n$
trivial cycles.

\begin{proof}
Let $\rho$ be an $n$ by $k$ rotation. That is,
$$\rho = (k \MOD{n}, k+1 \MOD{n}, \dots, k+n-1 \MOD{n})$$
Suppose that $\rho$ has a trivial cycle. This means that there is an
$0 \leq i < n$ such that $i = k + i \MOD{n} \Rightarrow 
k = 0 \MOD{n}$. Then, 
\begin{eqnarray*}
    \rho &=& (0 \MOD{n}, 1 \MOD{n}, \dots, n-1 \MOD{n}) \\
         &=& (0, 1, \dots, n-1)
\end{eqnarray*}
which means that $\rho$ does not move any element.
\end{proof}


\paragraph{Exercise 11.11}
How many assignments does 3-reverse rotate perform?

\begin{proof}[Answer]
Call $\alpha_r(f,m,l)$ the number of assignments we seek.
First, let $\sigma(f,l)$ be the number of swaps performed by \Rev{f}{l}. Since this
function swaps element $f$ with element $l-1$, element $f+1$ with element
$l-2$, and so on, we have that 
$$\sigma(f, l) = \floor{\frac{l-f}{2}}$$
Then, if we note the number of swaps performed by 3-reverse rotate by 
$\sigma_r(f,m,l)$,
\begin{eqnarray*}
\sigma_r(f,m,l) &=& \floor{\frac{m-f}{2}} + \floor{\frac{l-m}{2}} + 
                    \floor{\frac{l-f}{2}} \\
             &\leq& \frac{m-f}{2} + \frac{l-m}{2} + \frac{l-f}{2} \\
                &=& \frac{(m-f) + (l-m) + (l-f)}{2} \\ 
                &=& \frac{2l - 2f}{2} \\
                &=& l - f \\
                &=& n
\end{eqnarray*}
being $n$ the number of elements being rotated. Thus, at most $n$ swaps are 
performed by 3-reverse rotate (and at least $n-1$, following a similar
approach). Since each swap takes three assignments,
$$3(n-1) \leq \alpha_r(f,m,l) \leq 3n$$
\end{proof}
