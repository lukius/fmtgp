\documentclass[a4paper,10pt]{article}
\usepackage{multirow}
\usepackage{graphicx}
\usepackage{fancyhdr}
\usepackage{lastpage}
\usepackage{amssymb}
\usepackage{epsfig}
\usepackage{latexsym}
\usepackage{amstext}
\usepackage{amsmath}
\usepackage{amsfonts}
\usepackage{amsthm}
\usepackage{array}
\usepackage{tikz}
\usepackage{moreverb}
\usepackage[a4paper, total={7in, 10.5in}]{geometry}
\usepackage[T1]{fontenc}
\usepackage{gfsartemisia-euler}

\newcommand{\Nat}{\ensuremath{\mathbb{N}}}

\begin{document}
\pagestyle{fancyplain}
\cfoot{- \thepage/\pageref{LastPage} -}
\renewcommand{\headrulewidth}{0pt}

\paragraph{Exercise 4.4}
Prove that, for any odd square number $x$, there is an even square number $y$
such that $x+y$ is a square number.\\

\begin{proof}
Since $x$ is square and odd, there must be an $n \in \Nat$ such that
$x = (2n+1)^2$. Let $y$ be some even square number. Thus, there must be an
$m \in \Nat$ such that $y = (2m)^2$. It follows that
\begin{eqnarray*}
x+y &=& (2n+1)^2 + (2m)^2 \\
    &=& 4(n^2 + m^2) + 4n + 1 
\end{eqnarray*}
We must define $m$ as a function on $n$ in such a way that this number conforms
a square. In order to do this, let's see what happens for some small cases:
\begin{itemize}
    \item If $n = 1$, then $x = 9$. If we set $m = 2$, $y = 16$ and $x+y = 25$,
    which is a square.
    \item If $n = 2$, then $x = 25$. Taking $m = 6$, $y = 144$ and $x+y = 169$,
    which is a square (since $13^2 = 169$).
    \item If $n = 3$, then $x = 49$. Now, $m$ can be $12$, and then $y = 576$
    and $x+y = 625 = 25^2$.
\end{itemize}
A careful analysis of these cases reveals a pattern: $m = n^2 + n$. Substituting
this in the equation shown before,
\begin{eqnarray*}
x+y &=& 4(n^2 + m^2) + 4n + 1  \\
    &=& 4(n^2 + (n^2 + n)^2) + 4n + 1 \\
    &=& 4n^2 + 4(n^2 + n)^2 + 4n + 1 \\
    &=& 4(n^2 + n)^2 + 4(n^2 + n) + 1 \\
    &=& (2(n^2 + n) + 1)^2 \\
\end{eqnarray*}
\end{proof}

\end{document}
