\section*{Chapter 3: Ancient Greek Number Theory}

\paragraph{Exercise 3.6}
Prove that if $n$ and $m$ are coprime, then $\DivSum{nm} = 
\DivSum{n} \, \DivSum{m}$

\begin{proof}
Let the prime factorization of $nm$ be 
$nm = p_1^{\alpha_1} \dots p_k^{\alpha_k}$.
Being $\GCD{n}{m} = 1$, if $\Divides{p_i}{n}$ then $\NotDivides{p_i}{m}$
(and viceversa), $1 \leq i \leq k$. In consequence, if the prime factorization
of $n$ is $n = q_1^{\beta_1} \dots q_l^{\beta_l}$, then any $q_i$ cannot appear
in the prime factorization of $m$. That is, if the prime
factorization of $m$ is $m = r_1^{\gamma_1} \dots r_s^{\gamma_s}$, 
then $q_i \neq r_j$, $1 \leq i \leq l$, $1 \leq j \leq s$. Thus,
\begin{eqnarray*}
\DivSum{nm} &=& \DivSum{p_1^{\alpha_1} \dots p_k^{\alpha_k}} \\
            &=& \prod_{i = 1}^{k}{\frac{p_i^{\alpha_i + 1} - 1}{p_i - 1}} \\
            &=& \prod_{i = 1}^{l}{\frac{q_i^{\beta_i + 1} - 1}{q_i - 1}} \,
                \prod_{j = 1}^{s}{\frac{r_i^{\gamma_i + 1} - 1}{r_i - 1}} \\
            &=& \DivSum{q_1^{\beta_1} \dots q_l^{\beta_l}} \,
                \DivSum{r_1^{\gamma_1} \dots r_s^{\gamma_s}} \\
            &=& \DivSum{n} \, \DivSum{m}
\end{eqnarray*}
\end{proof}

\paragraph{Exercise 3.7}
Prove that every even perfect number is a triangular number.

\begin{proof}
Let $k$ be an even perfect number. Then, by the Euclid-Euler theorem,
$k = 2^{n-1} (2^n - 1)$ for some $n \in \Nat$, where $2^n - 1$ is prime.
Thus,
\begin{eqnarray*}
k &=& 2^{n-1} (2^n - 1)\\
  &=& (2^n - 1) (2^n / \, 2) \\
  &=& \frac{(2^n - 1) 2^n}{2} \\
  &=& \triangle_{2^n - 1}
\end{eqnarray*}

\end{proof}


\paragraph{Exercise 3.8}
Prove that the sum of the reciprocals of the divisors of a perfect number is
always 2.

\begin{proof}
Let $n$ be a perfect number with divisors $d_1,\dots,d_k$. By definition of
perfect number, we have that
$$\DivSum{n} = d_1 + \dots + d_k = 2n$$
which implies that 
$$2 = \frac{d_1 + \dots + d_k}{n} = \frac{d_1}{n} + \dots + \frac{d_k}{n}$$
Since $\Divides{d_i}{n}$, $1 \leq i \leq k$, $n = d_i q_i$. But
$\Divides{q_i}{n}$ as well, and so $q_i = d_j$. Then, $d_i / n = 1 / d_j$.
In consequence, every summand on the right-hand side of the previous equation
can be rewritten as the reciprocal of some divisor of $n$, and so
$$2 = \frac{d_1}{n} + \dots + \frac{d_k}{n} = \frac{1}{d_1} + \dots + \frac{1}{d_k}$$

\end{proof}
